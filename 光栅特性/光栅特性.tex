\documentclass{ctexart}
\usepackage{amsmath}
\usepackage{geometry}
\usepackage{hyperref}
\usepackage{float}
\usepackage{graphicx}
\title{光栅特性及测定光波波长}
\author{陈启钰\,\,2300011447}
\date{\today}
\begin{document}
	\maketitle
	\paragraph{光栅的衍射光谱}当使用平行光源照射光栅,并用会聚透镜将衍射后的平行光汇聚起来,在透镜的焦面上会出现按波长次序及谱线级次,自第0级开始左右两侧有短波向长波排列的各种颜色的谱线,称为光栅的衍射光谱。
	\paragraph{衍射光谱的特点}左右对称,从第0级开始左右两侧有短波向长波排列的各种颜色的谱线。
	\paragraph{角色散率}角色散率$D$的定义为同一级两条谱线衍射角之差$\Delta\phi$与他们的波长差$\Delta\lambda$之比。
	\begin{align}
		D\equiv\frac{\Delta \phi}{\Delta \lambda}
	\end{align}
	\paragraph{色分辨本领}色分辨本领$R$定义为两条刚好能够被该光栅分辨开的谱线的波长差$\Delta\lambda\equiv\lambda_2-\lambda_1$去除它们的平均波长$\bar{\lambda}$
	\begin{align}
		R\equiv\frac{\bar{\lambda}}{\Delta\lambda}
	\end{align}
	\paragraph{两者的区别}色分辨本领强调的是光栅分辨谱线的能力,而色分辨本领只是反映两条谱线中心分开的程度,不涉及它们是否能够分辨。
	\paragraph{光栅的调节}
	\subparagraph{调节光栅平面与平行光管光轴垂直}先用汞灯把平行光管的狭缝照亮,使望远镜目镜中分划板中心垂直线对准狭缝像。然后固定望远镜,把光栅放在载物台上,使光栅平面大致垂直于望远镜。再用自准直法调节使得从光栅平面反射回来的亮十字像与分划板MN线重合。再调节平行光管狭缝像与十字像重合,使光栅平面与平行光管光轴垂直。
	\subparagraph{调节光栅使其刻痕与仪器转轴平行}
	松开望远镜的紧固螺丝,转动望远镜,找到光栅的一级和二级衍射谱线,调节载物台使得各条谱线中点与分划板圆心重合,即使两边光谱等高。调好后,再返回来检查光栅平面是否仍然保持与平行光管光轴垂直,若有改变,则要反复调节直到两个条件均能满足。
\end{document}