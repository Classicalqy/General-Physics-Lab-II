\documentclass{ctexart}
\usepackage{amsmath}
\usepackage{geometry}
\usepackage{graphicx}
\usepackage{float}
\usepackage{hyperref}
\title{复摆实验}
\author{陈启钰\,\,2300011447}
\date{\today}
\begin{document}
	\maketitle
	\section{引入相对重心转动惯量和回转半径的意义}
	引入相对重心的转动惯量以后,复摆的转动惯量$I$可以表示成相对重心转动惯量$I_G$与$mh^2$之和,使得对复摆相对某点的转动惯量更加容易表示,同时方便线性拟合。引入回转半径$R_G$是引入了一个与$h$同量纲的量 ,方便方程的表示。
	\section{复摆的共轭性}
	假如一个复摆的振动中心在C点,悬点在O点(OCG三点共线)。如果这个摆绕过C点的新轴摆动且该轴平行于过O点的轴,则它的周期不变,O成了新的振动中心,这就是复摆的共轭性。
	
	由共轭性,有
	\begin{align}
		h^2-\frac{T^2}{4\pi^2}gh+R_G^2=0
	\end{align}
	可得
	\begin{align}
		h_1+h_2=\frac{T^2}{4\pi^2}g=L,g=\frac{4\pi^2(h_1+h_2)}{T^2}=\frac{4\pi^2L}{T^2}
	\end{align}
	实验中,只需要找到过质心的直线上两点,且分别以这两个点为悬点时复摆的振动周期相同,在测量出两个点的距离以及周期即可得到重力加速度。
	\section{支撑法安装复摆的优点}
	稳定性:支撑法是一种相对稳定的安装方法。通过将复摆的轴与支撑物(例如墙壁或天花板)连接,可以确保摆动过程中不会发生意外脱离或倾斜。
	减少振动干扰:支撑法可以减少外部振动对复摆的影响。当复摆悬挂在支撑物上时,支撑物可以吸收部分外部振动,从而减少了摆动的干扰。
	方便安装和调整:支撑法通常比其他安装方法更容易安装和调整。只需将复摆的轴与支撑物连接,而无需复杂的支架或其他设备。
	\section{测量中心到悬点的距离}
	将复摆的每个孔都当作悬点,并测量每个振动周期,对于第$n$个悬点
	\begin{align}
		T(n)=2\pi\sqrt{\frac{I_G+mr_n^2}{mgr_n}}
	\end{align}
	式中$r_n$为第$n$个孔到质心的距离,设质心在第$n_0$到第$n_0+1$个孔之间,且距离第$n_0$个孔的距离为$a$,则有
	\begin{align}
		r_n=|a+(n_0-n)d|
	\end{align}
	其中$d$是相邻两孔的距离。两个最小的周期值对应的$n$可以确定$n_0$,然后$a$可以通过求解方程求到(也可以通过拟合得到)。
	\section{周期测量的误差来源}
	在实际测量中,复摆的摆动角度有可能不满足小角度近似,会引起误差。\\
	空气浮力和阻力可能会对周期产生影响。\\
	复摆振动过程中还受到摩擦阻力等影响周期。\\
	光电计时器等测量精度也会影响周期测量。
	\section{实现周期微调的方案}
	可以通过调节复摆上的微调螺母对周期进行微调。\\
	此外,还可以通过安装加重片进行周期微调。
\end{document}